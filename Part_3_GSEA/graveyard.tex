\begin{frame}
  \frametitle{LFC Sensitivity Testing}

  \begin{itemize}
  \item Question: Are certain gene/microbe sets enriched at all \(\epsilon^\perp \in (-\infty, \infty)\)?
  \item Answer: Yes!
  \end{itemize}

  \vspace{15px}
  
  Consider a test based on the p-value
  \begin{align*}
    p = \sup_{\epsilon^\perp \in (-\infty, \infty)} p_{\epsilon^\perp}
  \end{align*}

  This test:
  \begin{enumerate}
    \item Controls Type-I error even when \(\epsilon^\perp\) is unknown
    \item Has demonstrated power in real dataset between 10-15\% 
  \end{enumerate}
\end{frame}

    \begin{tabular}{| c | c | c | c | c | c | c | c | c | c |}
      \hline
      \makecell{Gene \\Name} & RB1 & IL2 & APP & AR & HTT & IL6 & B2M & RGN & CAT \\
      \hline
      LFC & 2.8 & 2 & 1.5 & 0.7 & 0 & -0.6 & -1.1 & -2.1 & -3 \\
      \hline
      Set \(S\) & X & & X & X & X & & & & \\
      \hline
      Weights & .56 & -.2 & .3 & 0.14 & 0 & -.2 & -.2 & -.2 & -.2 \\
      \hline
      \makecell{Running \\ Sum} & .56 & .36 & .66 & .8 & .8 & .6 & .4 & .2 & 0 \\
      \hline
      \makecell{Enr. \\ Score} & & & & .8 & & & & & \\
      \hline
    \end{tabular}
\begin{frame}
  \frametitle{Scale and Sensitivity Analyses}

  \begin{exampleblock}{Key Points of this Workshop}
    \begin{enumerate}
      \item Sequence count data do not measure the scale
      \item Wrong assumptions about the scale can cause false positives
      \item We propose solutions to this problem
    \end{enumerate}
  \end{exampleblock}

  \begin{itemize}
    \item The previous presentation discussed a Bayesian solution
    \item Now we turn to a (Frequentist) sensitivity analysis solution
  \end{itemize}
\end{frame}
